\chapter*{ВВЕДЕНИЕ}
\addcontentsline{toc}{chapter}{ВВЕДЕНИЕ}


Вредоносное ПО может работать как приложение в пользовательском пространстве, так и как часть операционной системы. Руткиты чаще всего относятся ко второй категории, что дает им больше возможностей, делает их более опасными и максимально затрудняет их поиск и нейтрализацию.

Руткиты являются небольшими наборами инструментов, утилит и сценариев. Главной целью внедрения их в целевую систему является получение прав администратора, поэтому система может либо использоваться удаленно для сбора секретных данных, либо использоваться для проведения атак в отношении других уязвимых систем, внедрения руткита и получения доступа к ним.

Цель работы --- реализовать загружаемый модуль ядра, позволяющий скрывать присутствие пользователя в системе.

